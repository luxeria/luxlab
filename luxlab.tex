\documentclass[a4paper,
               10pt,
               fleqn]{article}

\author{Ervin Mazlagic}
\title{Paper Manual}

\usepackage{luxpaper}
\usepackage{luxtitle}
 
\begin{document}
\luxtitle{Papers}
         {LuXLab Richtlinien}
         {Ervin Mazlagi\'c}
         {Adligenswil}
         {2012}
         
\begin{abstract}
\noindent
Ab Sommer 2012 verfügt der Verein LuXeria einen Elektronik-Laborplatz
welcher intern als LuXLab bezeichnet wird. Das Unterhalten und benutzten 
des LuXLab erfordert gewisse Richtlinien welche den Betrieb als auch
die Sicherheit der Nutzer und Infrastruktur gewährleisten.
\end{abstract}

\newpage

\tableofcontents
\newpage

%%%%%%%%%%

\section{Nutzung}
Grundsätzlich gilt das Nutzungsrecht des Elektronik-Labors für sämtliche Mit-
glieder des LuXeria, welche sich mit den hier beschriebenen Richtlinien 
einverstanden erklären und die enstprechenden Leistungen entrichten 
(siehe Abschnitt Gebühren). Eine Ausnahme dieser Regel kann im Rahmen 
von unterstützten Projekten durch den Vorstand erfolgen.

\subsection{Zugang}
Die Vereinstreffen von LuXeria gelten als zeitlicher Nutzungsrahmen 
des LuXLab, welche jeweils Mittwochabend von 17:30 bis 22:30 Uhr 
stattfinden. Für Anwesenheiten ausserhalb dieser Zeiten, sind nur die sog. Vertrauenspersonen\footnote{Vertrauenspersonen sind aktive LuXeria-Mitglieder,
    welche dem VFI und somit dem Sicherheitsdienst (aktuell Securitas) des
    Ringier Print Areals kommuniziert und registriert werden mit den zum
    Zutritt erforderlichen Angaben.}
berechtigt bzw. Personen die in deren Begleitung erscheinen.

\subsection{Reservierungen}
Alle Personen die Anspruch auf die Nutzung des LuXLab besitzen, haben die
Möglichkeit, Laborzeiten zu reservieren. Hierfür wird der Vereinspräsident 
beauftragt bzw. angefragt die Reservation durchzuführen oder abzuweisen
(mit evtl. Verweis auf einen anderen Termin).

\section{Unterstützung}
Arbeiten welche im LuXLab getätigt werden, müssen der Idee\footnote{
     Als Idee ist gemeint, wie sich der Verein bzw. dessen Zweck definiert.
     Dies ist vollumfänglich in den Statuten des Vereins niedergeschrieben.}
des LuXeria entsprechen. 

Besonderer Wert wird dabei auf den \emph{nicht kommerziellen Charakter} des
Vereins gelegt. Wird innerhalb des LuXLab ein Ergebnis erreicht welches 
vermarktet werden kann, 
ob gezielt oder nicht und unabhängig vom Status der beteiligten Personen\footnote{
    Da Dritte das LuXLab nur durch ein aktives Mitglied des Vereins LuXeria
    benutzen können und dieses als Vertreter bzw. Vormund Dritter innerhalb des
    Vereins und somit des LuXLab gilt, gelten die sowohl die Statuten 
    als auch alle anderen Richtlinen des
    Vereins für alle entstandenen Inhalte und Ergebnisse von Dritten.}
, hat der Verein LuXeria das alleinige Vorzugsrecht auf die Vermarktung 
oder kann auf dieses Verzichten\footnote{Der Verein LuXeria hat das
    Recht die Inhalte, welche innerhalb des LuXLab entstehen unter
    eine dem Verein angebrachte Nutzung bzw. Lizenz (wie beispielsweise
    die GPL oder Creative Commons usw.) zu stellen und 
    es kann auf dieses Recht verzichtet werden.}.

\section{Sicherheit}
Das LuXLab ist nach gängigen Sicherheitsrichtlinien einzurichten und
allfällige Mängel oder Bedenken dazu schriftlich dem Vorstand von LuXeria
zu melden\footnote{
    Als schriftlicht gelten hierbei Briefe und E-Mails.}.
Zwingend für die Inbetriebnahme bzw. Nutzung des LuXLab ist das vorhandensein
eines Notschalters\footnote{
    Hier ist kein Industrie-Notschalter gemeint, sondern ein Schalter der
    in der Lage ist, alle Gerätschaften des LuXLab von der Netzspannung
    zu trennen.}
und eines CO$_2$ Feuerlöschers\footnote{
    Andere Feuerlöscher (z.B. Schaum- und Pulverlöscher) sind nicht für
    elektronische Arbeitsplätze geeignet und dürfen nicht ich Sichtweite
    zum LuXLab platziert werden.}.
    
Die Sicherheit ist grundsätzlich durch den jeweiligen Nutzer oder dessen
Vertreter zu gewärleisten.
Der Verein LuXeria entzieht sich jeglicher Verantwortung oder Haftung für 
Unfälle aller Art.

\section{Verantwortung}
Das LuXLab stellt einen eigenen Bereich bzw. Funktion innerhalb von LuXeria 
dar, welche von einem Mitglied wahrgenommen wird und somit die 
Verantwortung, Instandhaltung und Betreuung zugesprochen bekommt.

Dieser ist verantwortlich für die Instandhaltung des Arbeitsplatzes und der 
Geräte als auch für die Nutzer des Labors. Weiter ist dieser auch befugt, 
Nutzungsrechte mit sofortiger Wirkung abzusprechen, wobei solche
Fälle schriftlich und umgehend dem Vorsstand zu melden sind.

\subsection{Nutzer}
Jeder Nutzer des LuXLab trägt direkt die Verantwortung für sich,
die zur Verfügung stehenden Gerätschaften als auch das LuXLab als Ganzes.

\section{Instandhaltung}

\subsection{Inventar}
Über alle Geräte, Werkzeuge und Materialien wird laufend ein Inventar 
geführt, welches nach Möglichkeit den Besitzer, das Ein-/Ausführungsdatum 
und Zustand beschreibt. 
Dazu stehen die folgenden Kategorien zur Verfügung: VFI, LuXeria, Member und
Projektgruppen\footnote{
    Projektgruppen können auch Dritte, also Personen die keine Mitgliedschat
    von LuXeria besitzen, beinhalten oder rein durch Dritte belegt sein. 
    In letzerem Fall ist eine Vertretung durch ein Vereinsmitglied gegeben.}
.

\subsection{Logbuch}
Jeder Nutzer ist dazu verpflichtet sämtliche aktivitäten zu registrieren.
Dies erfolgt per Eintrag in ein \emph{Logbuch}. Dort sind zu folgenden
Punkten Einträge zu erstellen:
\begin{itemize}
    \item Benutzer (Person oder Gruppe)
    \item Probleme/Komplikationen in Zusammenhang mit Infrastruktur etc.
    \item Wünsche und Verbesserungsvorschläge
    \item Materialbezüge
\end{itemize}

\subsection{Projekte}
Jedes Projekt hat einen offizielen Namen und einen Besitzer\footnote{
    Eine Person, die bezüglich des Projektes angesprochen werden kann.}.
Zu jedem Projekt wird ein offenes Journal geführt, welches im Rahmen der
Nutzer des LuXLab und LuXeria eingesehen werden kann. 
Dieses soll die grobe Entwicklung
eines Projektes wiedergeben und muss keine Detailinformationen 
(wie Technologien oder Methoden) enthalten. 

Diese Angaben und Informationen werden genutzt um eine aktive Pflege des
LuXLab zu erleichtern und die Mitglieder von LuXeria über aktuelle
Ergebnisse und Entwicklungen zu informieren.

\subsection{Arbeitsplatz}
Jeder Nutzer steht in der Pflicht, beim Verlassen des LuXLab die
Ausgangslage bzw. eine abgesprochene Ordnung zu erstellen. Hierzu
werden von LuXeria \emph{Projektboxen} zur Verfügung gestellt. Diese
können und sollen benutzt werden um alle projektrelevanten Objekte 
so zusammenzutragen, damit das Erstellen einer Ordnung erleichtert wird.
 
\subsection{Geräte}
Die zur Verfügung gestellten Gerätschaften sind nach deren Benutzung in
den Ausgangszustand bzw. einen abgesprochenen Zustand zu versetzten.
Beim Verlassen\footnote{
    Hier ist längeres Verlassen gemeint. Das Verlassen den LuXLab in 
    Sinne kurzer Pausen usw. sind hiervon nicht betroffen.}
des LuXLab sind sämtliche Geräte von der Netzspannung zu trennen und 
Batteriebetribene Geräte abzustellen.

Defekte oder Unsicherheiten bezüglich des Zustandes der Gerätschaften 
sind dem Verantwortlichen des LuXLab umgehend zu melden.

\subsection{Verbrauchsmaterial}
Grundlagendes Verbrauchmaterial\footnote{
    Als grundlegendes Verbrauchsmaterial gelten Materialien der
    Lötstationen wie Drähte und Lötzinn aber auch Bauteile ohne
    speziellen Charakter (Widerstände, einfache Halbleiter usw. 
    die keine Auffälligkeit haben bezüglich ihrer Lieferbarkeit
    oder Stückpreisen).}
wird zur Verfügung gestellt vom Verantwortlichen des LuXLab bzw.
vom Verein LuXeria.

Grössere Projekte werden von der entsprechenden Projektgruppe
mit Material versorgt. Bezüge aus dem zur Verfügung gestellten 
Sortiment sind einzutragen im \emph{Führungsblatt}.

\section{Gebühren}
Nutzer des LuXLab haben eine jährliche Gebühr zu entrichten in
Höhe des Mitgliederbeitrages von LuXeria.
Die entrichtung dieser Grbühr fällt per Eintritt oder bei
bestehender Mitgliedschaft mit den Entrichten der
Mitgliederbeiträge an. Die Entrichtung der Gebühr verleiht
automatisch das Nutzungsrech bis zum Ende des Kalenderjahres
bzw. bis zum Termin der Entrichtung der Mitgliederbeiträge 
von LuXeria.

Die einbezahlten Gebühren werden vollumfänglich dazu benutzt, 
das LuXLab zu unterhalten. Überschüsse am Ende des Kalenderjahres 
werden erfasst und als Budget zur unterstützung von Projekten
freigegeben.

%%%%%%%%%%

\end{document}




































